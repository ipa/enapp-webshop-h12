\documentclass[11pt]{amsart}
\usepackage{geometry}                % See geometry.pdf to learn the layout options. There are lots.
\geometry{letterpaper}                   % ... or a4paper or a5paper or ... 
%\geometry{landscape}                % Activate for for rotated page geometry
%\usepackage[parfill]{parskip}    % Activate to begin paragraphs with an empty line rather than an indent
\usepackage{graphicx}
\usepackage{amssymb}
\usepackage{epstopdf}
\DeclareGraphicsRule{.tif}{png}{.png}{`convert #1 `dirname #1`/`basename #1 .tif`.png}


\begin{document}
% document options
\title{ENAPP 2012 - Webshop}
\author{Iwan Paolucci}
%
\maketitle
%
%\newpage{}
%%%%%%%%%%%%%%%%%%%%%%%%%%%%%%%%%
% Introduction
%%%%%%%%%%%%%%%%%%%%%%%%%%%%%%%%%
\section{Intro}
In the module ENAPP.H12 a Webshop has to be developed. This documentation covers the most neccesary information and a short overview of the architecture. \\
\subsection{}Technology\\
A short overview over the technologies used.
\begin{itemize}
\item J2EE 1.6 (Java Enterprise Edition)
\item Java 7
\item Java Server Faces (JSF)
\item Java Message Service (JMS)
\item SOAP Webservices
\item Rest Webservices
\item JDBC / JPA (MySQL)
\end{itemize}
%
\newpage{}
%%%%%%%%%%%%%%%%%%%%%%%%%%%%%%%%%
% Infrastructure
%%%%%%%%%%%%%%%%%%%%%%%%%%%%%%%%%
\section{Infrastructure} 
%
This section covers information about the Infrastructure. \\
%\subsection{}Development \\
%\textbf{Database} \\
%User: root \\
%Password: 123 \\
% 
\subsection{}Integration \\
Description of the integrationplattform in the EnterpriseLab @ HSLU.\\
\subsubsection{}OperatingSystem \\
Type: Solaris \\
User: tapaoluc (EL User) \\
root Password: ENAPP\_H12
%
\subsubsection{}Database \\
Type: MySQL \\
Server: s0160.intra015.el.campus.intern \\
User: enapp \\
Password: enapp 
%
\subsubsection{}Applicationserver \\
Type: Glassfish 3+ \\
Adminpanel: https://s0160.intra015.el.campus.intern:4848 \\
User: admin \\
Password: ENAPP\_H12 \\
%
\subsubsection{}/etc/hosts \\
For the connection to the Navision service an additional entry to the hosts file is needed.\
\begin{verbatim}
#navision
10.29.2.12      icompanydb01.icompany.intern
\end{verbatim}
%
\subsection{Configuration}
\subsubsection{}Database \\
\textit{JDBC Connection Pool}\\
Poolname: EnappWebshopTapaolucPool \\
Resource Type: javax.sql.DataBase \\
Datasource Classname: com.mysql.jdbc.jdbc2.optional.MysqlDataSource \\
% additional
\textit{Additional Properties on Connection Pool} \\
password = enapp \\
user = enapp \\
servername = s0160.intra015.el.campus.intern \\
roleName = com.mysql.jdbc.Driver \\
datasourceName = jdbc:mysql://s0160.intra015.el.campus.intern:3306/enappwebshop \\
databaseName = enappwebshop \\
portNumber = 3306 \\
%
\textit{Ressource} \\
JNDI Name: jdbc/enappwebshoptapaoluc \\
Poolname: EnappWebshopTapaolucPool \\
%
\subsubsection{}Java Message Service \\
\textit{Queuefactory} \\
Poolname: jms/purchasequeuefactory \\
Ressource Type: javax.jms.QueueConnectionFactory \\
Transaction: XATransaction \\
AdditionalProperty: AddressList = mq://10.29.3.152:7676/jms \\
\textit{Queue} \\
JNDI Name: jms/purchasequeue \\
Physical Name: EnappQueue \\
Resource Type: javax.jms.Queue \\
%
\subsubsection{}Security
%
\textit{Security - Realm} \\
Configuration: server-config \\
Realm Name: enappwebshoprealm \\
Classname: com.sun.enterprise.security.auth.realm.jdbc.JDBCRealm \\
JAAS Context: jdbcRealm \\
JNDI Name: jdbc/enappwebshoptapaoluc \\
Usertable: customer \\
Usercolumn: username \\
Passwordcolumn: password \\
Grouptable: customergroups \\
Grouptable-Usercolumn: username \\
Groupcolumn: groupname \\
Digest algorithm: none \\
Encryption algorithm: none \\
%
\textbf{Attention: There is no encryption so do not use your own passwords for tests!}
%
\subsubsection{}JNDI - Custom Ressources \\
This setting sets the stage of the application to production. \\
JNDI Name: javax.faces.PROJECT\_STAGE \\
Ressource Type: java.lang.String \\
Factory Class: org.glassfish.resources.custom.factory.PrimitivesAndStringFactory \\
Property: stage = production \\
%
\newpage{}
%%%%%%%%%%%%%%%%%%%%%%%%%%%%%%%%%
% Architectural overview
%%%%%%%%%%%%%%%%%%%%%%%%%%%%%%%%%
\section{Architecture}
\end{document}